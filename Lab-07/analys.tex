
\documentclass[12pt,a4paper]{article}

% ustawienia marginesu
\usepackage[left=1.6in,right=.8in,top=1.5in,bottom=1.5in]{geometry}

% polskie reguły dzielenia wyrazów itd
\usepackage{polski}

% polskie znaki zakodowane w UTF8
\usepackage[utf8]{inputenc}

% automatyczne generowanie odnośników w plikach PDF
\usepackage[pdftex,linkbordercolor={0 0.9 1}]{hyperref}

% pakiety matematyczne
\usepackage{amsthm,amsmath,amsfonts,amssymb,mathrsfs}

% ładne składanie odnośników do stron www
\usepackage{url}

% rozbudowane możliwości wypunktowań
\usepackage{enumerate}

% możliwość dodawania plików graficznych
\usepackage{graphicx} 

%%% definicje twierdzeń itd :)
\newtheorem{tw}{Twierdzenie}[section]
\newtheorem{stw}[tw]{Stwierdzenie}
\newtheorem{fakt}[tw]{Fakt}
\newtheorem{lemat}[tw]{Lemat}

\theoremstyle{definition}
\newtheorem{df}[tw]{Definicja}
\newtheorem{ex}[tw]{Przykład}
\newtheorem{uw}[tw]{Uwaga}
\newtheorem{wn}[tw]{Wniosek}
\newtheorem{zad}{Zadanie}

% oznaczenia zbiorów liczbowych
\DeclareMathOperator{\R}{\mathbb{R}}
\DeclareMathOperator{\Z}{\mathbb{Z}}
\DeclareMathOperator{\N}{\mathbb{N}}
\DeclareMathOperator{\Q}{\mathbb{Q}}


% wartość bezwzględna, norma, iloczyn skalarny, nośnik, rozpięcie przestrzeni...
\providecommand{\abs}[1]{\left\lvert#1\right\rvert}
\providecommand{\var}[1]{\operatorname{var}(#1)}

% fajne nagłówki i stopki na stronie
\usepackage{fancyhdr}
\pagestyle{fancy}
\fancyhf{}
\fancyfoot[R]{\textbf{\thepage}}
\fancyhead[L]{\small\sffamily \nouppercase{\leftmark}}
\renewcommand{\headrulewidth}{0.4pt} 
\renewcommand{\footrulewidth}{0.4pt}

% typowe dane dokumentu
\title{Funkcje ciągłe i różniczkowalne}
\date{\today}

% tu podaj swoje imię i nazwisko!
\author{Karol Chołaszczyński}

% zaczynamy dokument
\begin{document}
 
\maketitle

\tableofcontents

\section{Funkcje ciągłe} 

\begin{df}
(funkcja ciągła). Niech 
$f:(a,b) \to \mathbb{R},$
oraz niech
$x_{0}\in$(a,b). Mówimy, że funkcja $f$ jest ciągła w punkcie $x_{0}$ wtedy i tylko wtedy gdy:

\begin{center}
$\forall_{\varepsilon>0}\exists_{\delta>0} \forall x\in(a,b)\ |x-x_{0}|<\delta\leftarrow|f(x)-f(x_{0})|<\varepsilon.$ 
\end{center}
\end{df}

\begin{ex}
Wielomiany, funkcje trygonometryczne, wykładnicze, logarytmiczne są ciągłe w każdym punkcie swej dziedziny.
\end{ex}

\begin{ex}
Funkcja $f$ dana wzorem:
\begin{center}
\begin{displaymath}
f(x)= \left\{ \begin{array}{ll}
x+1 & \textrm{dla $x \neq 0$}\\
0 & \textrm{dla $x\ =\ 0$}\\
\end{array} \right.
\end{displaymath}
\end{center}
Jest ciągła w każdym punkcie poza $x_0$.
\end{ex}

Niech $\mathbb{Q}$ oznacza zbiór wszystkich liczb wymiernych.

\begin{ex}
Funkcja $f$ dana wzorem:
\begin{center}
\begin{displaymath}
f(x)= \left\{ \begin{array}{ll}
0 & \textrm{dla $x \in \mathbb{Q}$}\\
1 & \textrm{dla $x\ \notin\ \mathbb{Q}$}\\
\end{array} \right.
\end{displaymath}
\end{center}
Nie jest ciągła w żadnym punkcie.
\end{ex}

\begin{ex}
Funkcja $f$ dana wzorem:
\begin{center}
\begin{displaymath}
f(x)= \left\{ \begin{array}{ll}
0 & \textrm{dla $x \in \mathbb{Q}$}\\
x & \textrm{dla $x\ \notin\ \mathbb{Q}$}\\
\end{array} \right.
\end{displaymath}
\end{center}
jest ciągła w punkcie $x_{0}=0$, ale nie jest ciągła w pozostałych punktach dziedziny.
\end{ex}

\begin{zad}
Udowodnij prawdziwość podanych przykładów.
\end{zad}
\begin{df}
Jeśli funkcja 
$f:A \to \mathbb{R}$ jest ciągła w każdym punkcie swojej dziedziny A to mówimy krótko, że jest ciągła.
\end{df}
\begin{center}
Poniższe twierdzenia zbiera podstawowe własności zbioru funkcji ciągłych
\end{center}
\begin{tw}
Niech funkcje f,g: $R \to\mathbb{R}$ będą ciągłe, oraz niech $\alpha ,\beta \in\mathbb{R}$.Wtedy funkcje:
\\a) $h_{1}(x)=\alpha \cdot f(x)+ \beta \cdot g(x),$
\\b) $h_{2}(x)=f(x) \cdot g(x),$
\\c) $h_{3}(x)=\frac{f(x)}{g(x)}$ (o ile $g(x) \neq 0$ dla dowolnego $x\in\mathbb{R}$),
\\d) $h_{4}(x)=f(g)x))$,
\newline są ciągłe.
\end{tw}
Następne twierdzenie zwane powszechnie „własnością Darboux” lub twierdzeniem
o wartości pośredniej ma liczne praktyczne zastosowania. Mówi ono o tym,
ze jeśli funkcja ciągła przyjmuje jakieś dwie wartości, to przy odpowiednich założeniach
co do dziedziny, przyjmuje tez wszystkie wartości pośrednie. Możemy sobie to
łatwo wyobrazić na przykładzie funkcji, która opisuje zmianę temperatury w czasie. Jeśli o 7:00 było -1$^{\circ}$C a o 9:00 było 2$^{\circ}$C, to zapewne pomiędzy 7:00 a 9:00 był taki moment, że temperatura wynosiła dokładnie 0$^{\circ}$C.
\begin{tw}
Niech $f:[a,b]\to\mathbb{R}$ ciągła, oraz niech $f(a)\neq f(b)$. Wtedy dla dowolnego $y_{0}\in conv\{f(a),f(b)\}$ istnieje $x_{0}\in[a,b]$takie, że $f(x_{0})=y_{0}$.
\end{tw}

\section{Różniczkowalność}

\begin{df}
Niech $f: (a,b)\to\mathbb{R},x_{0}\in(a,b)$ oraz $f$ ciągła w otoczeniu punktu $x_{0}$. Jeśli istnieje granica:
\begin{center}
$\lim_{x \rightarrow x_{0}}
\frac{f(x)-f(x_{0})}{x-x_{0}}$
\end{center}
i jest skończona, to oznaczamy ją przez $f'(x_{0})$ i nazywamy pochodną funkcji $f$ w punkcie $x_{0}$.
\end{df}
\begin{df}
Jeśli funkcja $f$ posiada pochodną w kazdym punkcie swojej dziedziny, to mówimy, ze $f$ jest różniczkowalna. Istnieje wtedy funkcja $f'$, która każdemu punktowi z dziedziny funkcji $f$ przyporządkowuje wartość pochodnej funkcji $f$ w tym punkcie.
\end{df}
\begin{ex}
Wielomiany, funkcje trygonometryczne, wykładnicze, logarytmiczne są różniczkowalne w każdym punkcie dziedziny.
\end{ex}
\begin{ex}
Funkcja $f(x)=|x|$ jest stała, ale nie posiada pochodnej w punkcie $x_{0}$.
\end{ex}
\begin{tw}
Niech $f:[a,b]\to\mathbb{R}$ ciągła i różniczkowalna na (a,b). Dodatkowo niech $f'(x)\neq 0$ dla $x\in (a,b)$, oraz niech $m = min_{x\in[a,b]}f(x)$. Wtedy na pewno $f(a)=m, f(b)=M$ lub $f(a)=M, f(b)=m$.
\end{tw}
\begin{center}
I $\heartsuit$ \LaTeX
\end{center}

Kontakt: 
\author{Karol Chołaszczyński $<$\href{mailto:dr.judym44@gmail.com}%
{dr.judym44@gmail.com}$>$

\end{document}
